% kisegítő parancsok
\import{.}{csomagok}
\import{.}{szuro}

% amikor nincs semmilyen alap tag tag-elve, az azt jelenti, hogy mindent szeretnénk, hogy bekerüljön
\untagged{tetelek,bizonyitasok,kovetkezmenyek,megjegyzesek,peldak,magyarazatok,algoritmusok,fogalomtar}{
    \usetag{minden,tetelek,bizonyitasok,kovetkezmenyek,megjegyzesek,peldak,magyarazatok,algoritmusok,fogalomtar}
}

\tagged{1,2,3,4,5}{\usetag{filterelve}}

\tagged{minden,tetelek,bizonyitasok,kovetkezmenyek}{
    \usetag{tartalomjegyzek}
}

\newcommand{\KoTaF}{
    K{\smaller[1]{ö}}T\textsubscript{a}F
}

\newcommand{\tfh}{tegyük fel, hogy}
\newcommand{\Tfh}{Tegyük fel, hogy}
\newcommand{\acsa}{akkor és csak akkor}
\newcommand{\Acsa}{Akkor és csak akkor}

\newcommand{\rang}{\mathrm{rang}}

\newcommand{\cimHelper}[2]{
    \tagged{#1}{\iftagged{minden}{}{
        \begin{center}
            {\large \bfseries #2}
        \end{center}
    }}
}

\newcommand{\cim}[2][]{
    \begin{titlepage}
    \begin{center}
        {\huge \bfseries #2}
    \end{center}

    \cimHelper{tetelek}{tételek}
    \cimHelper{bizonyitasok}{bizonyítások}
    \cimHelper{kovetkezmenyek}{bizonyítások következményei}
    \cimHelper{fogalomtar}{fogalomtár}

    \begin{center}
        \vspace{8\baselineskip}
        {\larger #1}
    \end{center}

    \end{titlepage}
}

\newcommand{\abra}[3]{
   % \tagged{minden,abrak}{
    \begin{figure}[#3]
        \centering
        \subimport{.}{#1}
        \caption{#2}
        \label{abra:#1}
    \end{figure}
%    }
}

\newcommand{\fogalomimport}[1]{
    \par
    \noindent
    \textbf{\Glstext{#1}:}
    \glsdesc*{#1}
}

\newcommand{\fogalomprint}{
  \printunsrtglossary*{
    \renewcommand{\glossarysection}[2][]{}
  }
}

\renewcommand{\glsnamefont}[1]{\makefirstuc{#1}}
\newglossarystyle{mystyle}{%
\setglossarystyle{list}% base this style on the list style
\renewcommand*{\glossentry}[2]{%
\item[\glsentryitem{##1}%
  \glstarget{##1}{\glossentryname{##1}:}]%
  \noindent
  \glossentrydesc{##1}\glspostdescription\space ##2}
}
\setglossarystyle{mystyle}


% \theoremstyle{definition}
\newtheorem{tetelenv}{Tétel}[section]
\newcounter{bizcount}
\newtheorem*{bizenv}{Bizonyítás}
\newtheorem{lemmaenv}[tetelenv]{Lemma}
\newtheorem{kovenv}{Következmény}[tetelenv]
\newtheorem{allenv}[tetelenv]{Állítás}
\newtheorem{megjenv}{Megjegyzés}[tetelenv]
\newtheorem{peldaenv}{Példa}[tetelenv]
\newtheorem{magyarazatenv}{Magyarázat}[tetelenv]
\newtheorem{algenv}{Algoritmus}[section]

\newcommand{\tetel}[2][]{
    \iftagged{tetelek}{
        \begin{tetelenv}[#1]
            #2
        \end{tetelenv}
    }{\stepcounter{tetelenv}}
}

\newcommand{\biz}[1]{
    \tagged{bizonyitasok}{
        \begin{bizenv}
            #1
        \end{bizenv}
        \stepcounter{bizcount}
    }
}

\newcommand{\lem}[1]{
    \iftagged{tetelek}{
        \begin{lemmaenv}
            #1
        \end{lemmaenv}
    }{\stepcounter{tetelenv}}
}

\newcommand{\kov}[2][]{
    \tagged{kovetkezmenyek}{
        \begin{kovenv}[#1]
            #2
        \end{kovenv}
    }
}

\newcommand{\all}[2][]{
    \iftagged{tetelek}{
        \begin{allenv}[#1]
            #2
        \end{allenv}
    }{\stepcounter{tetelenv}}
}

\newcommand{\megj}[2][]{
    \iftagged{megjegyzesek}{
        \begin{megjenv}[#1]
            #2
        \end{megjenv}
    }{\stepcounter{megjenv}}
}

\newcommand{\pelda}[2][]{
    \iftagged{peldak}{
        \begin{peldaenv}[#1]
            #2
        \end{peldaenv}
    }{\stepcounter{peldaenv}}
}

\newcommand{\magyarazat}[2][]{
    \iftagged{magyarazatok}{
        \begin{magyarazatenv}[#1]
            #2
        \end{magyarazatenv}
    }{\stepcounter{magyarazatenv}}
}

\newcommand{\algoritmus}[2][]{
    \iftagged{algoritmusok}{
        \begin{algenv}[#1]
            #2
        \end{algenv}
    }{\stepcounter{algenv}}
}

\newcommand{\fejezetimport}{
    \foreach \a/\b in \fejezetek {
        \tagged{minden}{\untagged{filterelve}{\clearpage}}
        \usetag{f_\a}
        \import{.}{szuro}
        \iftagged{f_\a}{
            \tagged{minden,tetelek,bizonyitasok,kovetkezmenyek}{
                \section{\b}
                \import{fejezetek/\a}{main}
            }
        }{
            \stepcounter{section}
        }
    }
}

\newcommand{\fogalom}[3]{
    \longnewglossaryentry{#1}{name={#2}}{#3}
}

\foreach \a/\b in \fejezetek {
    \IfFileExists{fejezetek/\a/fogalmak.tex}{\import{fejezetek/\a}{fogalmak}}{}
}
\IfFileExists{fogalmak}{\import{.}{fogalmak}}{}
\tagged{minden,fogalmak,fogalomtar}{\makeglossaries}

\newcommand{\tartalomjegyzek}{
    \tagged{tartalomjegyzek}{
    \tableofcontents
    \clearpage
    }
}

\newcommand{\p}[2][1]{
    \iftagged{filterelve}{
        \tagged{#1}{#2}
    }{
        \tagged{minden}{#2}
    }
}

\newcommand{\cimke}[2]{
    \usetag{#1}
    \import{.}{szuro.tex}
    \tagged{#1}{#2}
}

\newcommand{\fogalomtar}{
    \tagged{fogalomtar}{
        \clearpage
        
        \tagged{tartalomjegyzek}{
        \section*{Fogalomtár}
        \addcontentsline{toc}{section}{Fogalomtár}
        }
        
        \fogalomprint
    }
}

