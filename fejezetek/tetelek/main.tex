\p{Itt megmutatjuk, hogy a tételeket és azok bizonyításait hogyan struktúráljuk. Ezt az elrendezést Vágó Tanárnő Analízis jegyzetei ihlették, ennek tiszteletére választottuk, köszönjük. 
Egy tételt így definiáluk:}
\begin{verbatim}
    \tetel[A tétel neve]{Ez a tétel leírása}
\end{verbatim}
\p[2]{
A tételnek opcionálisan adhatunk nevet, de általában nem szokott lenni. A példa így néz ki:
}

\tetel[A tétel neve]{
    Ez a tétel leírása
}

\p[3]{A tétel után definiálhatunk egy bizonyítást (akár többet is):}
\begin{verbatim}
    \biz{Ez a tétel bizonyítása}
\end{verbatim}

\biz{
    Ez a tétel bizonyítása
}

\p[4]{A tétel előtt lehetnek lemmák is:}
\begin{verbatim}
    \lem{Ez egy lemma}
\end{verbatim}

\lem{Ez egy lemma}

\p[5]{A tételnek lehet következménye (szintén lehet több is):}
\begin{verbatim}
    \kov{Ez egy következménye a tételnek}
\end{verbatim}

\kov{
    Ez egy következménye a tételnek
}

\p[5]{Tételek mellett léteznek állítások is, amiknek ugyanúgy lehet bizonyítása, következménye:}
\begin{verbatim}
    \all{Ez egy állítás, ugyanúgy működik mint egy tétel}
    \kov{Az állítás következménye, jól követi a számozást}
\end{verbatim}

\all{
    Ez egy állítás, ugyanúgy működik mint egy tétel
}
\kov{
    Az állítás következménye, jól követi a számozást
}

\p[5]{
Ezeken kívül használhatóak még: (a tétel szintaktikájával megegyeznek)}
\begin{itemize}
    \item megjegyzések - \verb-\megj-
    \item példák - \verb-\pelda-
    \item magyarázatok - \verb-\magyarazat-
    \item algoritmusok - \verb-\algoritmus-
\end{itemize}
\p[5]{
A következő fejezetben látni fogjuk hogy igény szerint egy tételgyűjteményt is tudunk automatikusan generálni például a szigorlatra készülőknek.
Ilyenkor a dokumentumban csak a tételek maradnak, minden más ami magyarázat, kimarad.}