\p[5]{Itt jön egy izgalmas dolog. Mint tudjuk, a matematikában vannak definíciók, tételek, lemmák, bizonyítások, stb... Ezek közül a definíciókat szeretnénk összegyűjteni egy fogalomtárba, de a dokumentum írása közben is szeretnénk elhelyezni rendre a hozzá tartozó fejezetbe is. 

A fogalmak kezelésére a \textbf{glossary} package-t használjuk. Készítettünk segítő parancsokat az könnyű kezelés érdekében:}
\begin{itemize}
    \item fogalom: megadjuk a fogalom típusát, majd egy azonosítót, és a fogalom szövegét, pl:
        \begin{verbatim}
            \fogalom{azonosito}{fogalom neve}
                {Ez a fogalom szövege}
        \end{verbatim}
    \item fogalomimport: ezzel lehet beszúrni egy adott fogalmat a szövegkörnyezetbe:
        \begin{verbatim}
            \fogalomimport{azonosito}
        \end{verbatim}
    \item fogalomprint: ez a parancs az összes fogalmat kilistázza, megjelenési sorrendben
        \begin{verbatim}
            \fogalomprint
        \end{verbatim}
        Ez a parancs annyira nem fontos, mivel általában csak a fogalomtárban lesz használva, de jó tudni róla.
\end{itemize}

\p[5]{
A fogalom parancsokat kérjük hogy a fejezethez tartozó, elkülönített fogalmat.tex-be tegyük, mert így szebben rendszerezhető és egységes marad.

Itt egy példa egy beszúrt fogalomra:
\fogalomimport{valami}
Ez a fogalom utáni szöveg. Ez itt egy hivatkozás a \gls{valami} fogalomra.}