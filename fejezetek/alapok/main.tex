\p[5]{
\cimke{valami}{Ez itt az első fejezet, egy ilyenben lehet kifejteni egyes témaköröket részletesen.
Minden ilyen számozott fejezet automatikusan bekerül a dokumentumba egyetlen paranccsal. A későbbiekben
megmutatjuk hogyan lehet feltételekkel megszabni, milyen részletesen kerüljön be a végleges doksiba.}

\cimke{masik}{Mindezt azért tesszük, hogy az átláthatóságot és a kezelhetőséget fenn tudjuk tartani.
Ha elég jól el vannak különítve a fájlok, akkor könnyedén lehet párhuzamosan
dolgozni többmindenen, ez a cél. Egy \KoTaF projekt mappa-elrendezését a következő módon követjük:}
\begin{itemize}
    \item main.tex - a fő fordítási egység forrása
    \item packagek.tex - minden használt package-t itt definiálunk
    \item parancsok.tex - minden saját \LaTeX\ parancsot itt definiálunk
    \item fogalmak.tex - a dokumentumhoz tartozó fogalmak. A későbbiekben megmutatjuk, hogyan tudjuk importálni a szövegkörnyezetbe.
    \item fejezetek
        \begin{itemize}
            \item fejezet\_azonosito - ez egy fejezet mappája, a mappa neve a fejezet rövid azonosítható nevevével ellátva. Ezzel az azonosítóval tudjuk importálni a dokumentumba
                \begin{itemize}
                    \item main.tex - a fejezet forrása, itt a szerkesztőre van bízva, hogy tovább bontsa több al-fájlra vagy akár mappára igény szerint. Érdemes úgy elképzelni, mint egy önálló projekt ezen belül.
                    \item ...
                \end{itemize}
            \item ...
        \end{itemize}
\end{itemize}
}
\p[5]{
Ide, a fejezet forrásába továbbá importálhatunk:
\begin{itemize}
    \item további al-fejezeteket
    \item ábrákat
    \item bármi mást, amit a téma igényel
\end{itemize}

Egy fejezetet a fő main.tex-be a következőképpen tudunk importálni.
A preambulum részében van egy lista, oda tudunk felvenni újabb fejezeteket, mely
elrendezése a következő:
}
\begin{verbatim}
    \def\fejezetek{
        fejezet_mappa_neve/Fejezet címe,
        ...
    }
\end{verbatim}
\p[5]{Először a fejezet azonosítóját adjuk meg, perjellel elválasztva pedig a fejezet formázott címét tudjuk megadni. Mivel vesszővel vannak az elemek elválasztva, és ha vesszőt akarunk a címbe tenni, a
Ez azért szükséges, mert látni fogjuk, hogy dinamikusan lehet kiválasztani, hogy a forrás mely részei kerüljenek be a végleges dokumentumba.}