\p[5]{A jelenleginél valószínűleg bonyolultabb hierarchia fog keletkezni, ezért itt megmutatjuk mi a cél, ha több alfejezet lenne.

Alfejezetet célszerű külön forrásba helyezni, majd importálni:}
\begin{verbatim}
    \subimport{.}{cucc}
\end{verbatim}
\p[5]{
A pont a jelenlegi forrás mappáját jelöli, a második argumentum a forrás fálj nevét jelenti.
Eredménye:}

\subimport{.}{cucc}

\subsection{Másik alfejezet}

\p[5]{Nem muszáj külön fájlba tenni ezeket. A tartalomtól függ, hogy mennyire átlátható külön vagy egyben. Ha az ellenkező történne, itt nyugodtan lehet almappákat kialakítani szintúgy, mint az ábrákkal.}

\p[5]{Egy fontos megemlítendő, hogy a fejezet szövegét mindig helyezzük egy} \verb-\p{...}-
\p[5]{parancsba, ezzel szabályozzuk a láthatóságot a dokumentum generálásakor. Ha nem tennénk minden magyarázó szöveget egy ilyenbe, akkor minden esetben megjelenne, emiatt is kérjük a szerkesztők különös figyelmét erre.}

\p[5]{Ezt úgy érdemes elképzelni, mint egy paragrafus (p). Ami nem oda tartozik (mint például a fogalmak, tételek) azt ne tegyük paragrafusba. Az ábrák a megértést segítik, ezért azok belerartoznak a paragrafusok közé.}