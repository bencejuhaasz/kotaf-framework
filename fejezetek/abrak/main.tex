\p[5]{Az ábrákat a \textbf{tikz} package-el rajzoljuk, gráfokat pedig \textbf{pgfplots}-al.
Minden ábra kap egy saját fájlt lehetőleg az adott fejezet mappáján belül.
Itt egy ábra, demonstráljuk hogy importáljuk könnyedén:

}
\begin{verbatim}
    \abra{abra1}{Ez egy ábra}{h!}
\end{verbatim}
\p[5]{
Az első argumentum az ábra azonosítója, amire \textit{abra:...}-al lehet hivatkozni.
Az ábra forrásfájlja is ezt a nevet kell, hogy felvegye.
A második az ábra alatti leíró szöveg, a harmadik az elhelyezést paraméterezi.
Az eredmény:

\abra{abra1}{Ez egy ábra}{h!}

Minden ilyet, ami a fejezethez tartozik, a saját mappájába tegyük, értelemszerűen.
Ha kell, kialakíthatunk almappákat. Ez pedig egy hivatkozás \aref{abra:abra1}. ábrára szokásos módon az}
\verb-\aref{abra:abra1}-
\p[5]{paranccsal.}